\documentclass[11pt, titlepage, fleqn]{report}
\usepackage[utf8]{inputenc}
\usepackage[T1]{fontenc}
\usepackage{graphicx}
\usepackage{hyperref}
\usepackage{listings}
\usepackage{siunitx}
\usepackage[left=3cm, right=2cm, top=3cm, bottom=2cm]{geometry}
\usepackage{hsellogo}
\usepackage{hselfonts}
\usepackage{hselcolor}
\usepackage{hselmath}
\usepackage{amsmath}
\usepackage{siunitx}
\usepackage{parskip}
\usepackage{csquotes}
\usepackage{acronym}
\usepackage{wrapfig}
\usepackage{subcaption}
\usepackage{float}
\usepackage{hyphenat}
\usepackage{microtype}
\usepackage[ngerman, german]{babel}
\usepackage[citestyle=numeric,bibstyle=numeric, sorting = none,
hyperref=true,backref=true,maxcitenames=3,url=true,backend=biber,natbib=true]{biblatex}
\addbibresource{references.bib}
\author{Liebenow, Wozasek}
\date{\textit{<2020-01-26 Sun>}}
\title{Dokumentation HoloOSCv2\\\medskip
\large Dokumentation HoloOSCv2}
\hbadness 1000
\tolerance = 200

\begin{document}
\begin{titlepage}% Deckblattk
	\hsellogo\hfill Projektarbeit % etwa Bachelorarbeit, Masterarbeit
	\par
	\vspace{4cm}
	\noindent\parbox{0.8\textwidth}{\Huge RotaCon - Dokumentation}  
	\vspace{2cm}

	\Large \noindent vorgelegt von:
	\begin{itemize}
		\item Tino Liebenow - Matrikelnummer 7011830
	\end{itemize}
	\vspace{2cm}
	betreut duch\newline
	Dipl.-Ing. (FH) Jörg Strick\newline
	Abgabedatum: 30.01.2020
\end{titlepage}
	\newpage
	\tableofcontents
	\listoffigures% Abbildungsverzeichnis
    \newpage
    \section*{\Huge Abkürzungsverzeichnis}% Abkürzungen
    \label{sec:Abkürzungsverzeichnis}
    \vspace{1cm}
    \begin{acronym}
        \acro{hsel}[HSEL]{Hochschule Emden Leer}
    \end{acronym}
	\newpage
	\chapter{Prolog}
		Hier wird Tino am Ende der Arbeit eine tolle Einleitung hinzaubern, die dei Ausgangslage zu derzeitigen Aktivlautsprechern
		beschreibt, ähnlich der Idee. Außerdem schreibe ich in männlicher Form etc...\newline
		Außerdem reden wir immer vom AKTIV-Lautsprecher...
	\section{Überblick}
		Überblick über den Aufbau der Dokumentation.\newline
		Diese Dokumentation ist in vier Hauptteile gegliedert. Im Prolog wird die Ausgangssituation geschildert, sowie die Kerngedanken zum RotaCon.
	\section{Ziel}
		was möchte Tino heute eigentlich machen?\newline
		Ziel ist es ein Gerät zur Fernbedienbarkeit von Potentiometern, speziell an Lautsprechern, zu entwickeln. Dabei sollen weder Veränderungen
		am Lautsprechergehäuse, oder an der Technik vorgenommen werden.
		Zusätzlich ist eine Diskretisierung der Pegelwerte und deren Visualisierung via LCD-Display angedacht, sodass der Benutzer sichtbare 
		Werte zur Orientierung der aktuellen Einstellung bekommt. 
	\section{Stand der Technik}	
		gibts sowas schon?\newline
		Fernsteuerungen für Pegeleinstellungen sind nicht neu auf dem Markt. Dabei sind zwei Varianten verbreitet, zum Einen besteht die Möglichkeit,
		den Potentiometer auf der Platine des Lautsprechers auszutauschen, zum Anderen die Integrierung eines Controllers in den Signalweg.
		Die erste Variante setzt somit die Öffnung des Lautsprechergehäuses sowie Manipulation der Elektronik vorraus. Dies ist nicht nur für den durchschnittlichen
		Lautsprecherbesitzer ein schwieriger Eingriff, sondern hat ebenfalls Einfluss auf diverse Garantieansprüche.\newline
		Die andere Variante beinhaltet keine Veränderungen am Lautsprecher selbst, regelt jedoch nur die Signalpegel zum Lautsprecher hin und nicht den 
		integrierten Verstärker. Somit muss für diese Version der Verstärker stets auf Maximum geregelt sein, dies hat je nach Verstärker Einflüsse
		auf Soundqualität und Stromverbrauch.
	\chapter{Theorie-Teil}
		Das folgende Kapitel beinhaltet die theoretischen Bestandteile zur Umsetzung der Idee. Diese bilden in ihrer Gesamtheit das Konzept,
		nachdem der Prototyp entwickelt wird.
		\section{Theorie Hardware}
		\label{sec:Theorie Hardware}
			\sloppy \nohyphens{
			Lautstärkeregler sind in den meisten Fällen Potentiometer oder Inkrementalgeber% [https://de.wikipedia.org/wiki/Lautst%C3%A4rkeregler]
			Diese besitzen in der Regel eine 6mm oder 6.35mm Achse, in D-Form oder geriffelt.
			Um diese Achse rotieren zu können muss also ein Motor mit einer passenden Kupplung angebracht werden. Für präzise Kontrolle und ausreichendes Drehmoment 
			ist ein Schrittmotor geeignet. Diese gibt es in verschiedenen Ausfertigungen: Unipolar und Bipolar. 
			Unipolare Schrittmotoren haben vier Phasen und werden nur in einer Richtung von Strom durchflossen. Bipolare polen ihre Magnetfelder durch 
			Umkehrung der Stromrichtung um, sie haben zwei Phasen, erreichen ein höheres Drehmoment und sind durch ihre Funktionsweise etwas aufwendiger anzusteuern. 
			Für eine Schrittmotorsteuerung sind drei Funktionseinheiten notwendig. Diese bestehen aus Mikrocontroller, Steuerschaltung und Treiberstufen.
			Für dieses Projekt ist ein unipolarer Schrittmotor ausreichend, da keine hohen Drehmomente erreicht werden müssen. Dies hat den Vorteil, dass 
			die Steuerschaltung rein auf Softwarebasis gelegt werden kann. Als Treiber soll eine Darlington-Schaltung die Steuersignale für den Motor verstärken. 
			Die softwaretechnische Besteuerung ist durch ein Arduino mit integriertem Mikrocontroller angestrebt, dessen Spannungsversorgung von 5V 
			idealer Weise auch für die anderen Komponenten ausreichen soll, sodass ein Betrieb des Gerätes durch USB-Spannungsversorgung ermöglicht wird. 
			Da analoge Drehregler durch eine Minimal- und Maximalstellung begrenzt sind, ist auch der Antrieb des Motors in diesen Bereichen zu stoppen. 
			In erster Linie sollen Endschalter den Potentiometeranschlag erkennen und software-gesteuert die Rotation einstellen. Redundant dazu wird der Strohmverbrauch 
			des Schrittmotors durch die Kombination eines nicht-invertierenden Verstäkers und dem Analog-Digital-Wandlers vom Mikrocontroller überwacht. 
			Für die Bedienung des Gerätes soll ein Infrarotempfänger mit passender Fernbedienung im Frequenzbereich verwendet werden. Da sich Lautstärkeregler 
			meist nicht sichtbar am Lautsprecher befinden, ein IR-Empfänger jedoch eine quasioptische verbindung vorraussetzt, bieten sich zwei getrennte 
			Gehäuseteile an, die durch ein Kabel verbunden sind. In dem sichtbaren Gehäuse befindet sich ebenfalls das Display.}
		\newpage
		\section{Theorie Software}
		\label{sec:Theorie Software}
			Softwaretechnisch sind drei Bereiche abzudecken, genannt seien die Programmierung des Mikrocontrollers an sich, das Entwerfen eines Platinenlayouts 
			sowie der Entwurf der Gehäuse für den 3D-Druck.
			Bei der Programmierung des Mikrocontrollers soll beachtet werden, dass die aktuellen Werte der Motorstellung und der dazugehörigen Lautstärkeanzeige
			gespeichert werden, damit bei jeder Inbetriebnahme die jeweils vorherigen Werte beibehalten bleiben. Die einzige Schnittstelle zum 
			Benutzer ist das Display, daher müssen Fehler, Warnungen und wichtige Informationen auf dem Display abgebildet werden. Alle variablen Funktionen des Gerätes 
			sollen über die Fernbedienung steuerbar sein, also die Steuerung der Lautstärke, aber auch das Aktivieren, bzw. Deaktivieren der Displaybeleuchtung.
		\section{Theorie Mechanik}
			Die Übertragung des Drehmoments vom Motor an den Drehregler des Lautsprechers kann nur stattfinden, wenn der Motor starr gelagert ist. Daher ist das Gehäuse 
			mit Motor und Motorsteuerung zu fixieren. Weiterhin sind Lautstärkeregler an verschiedenen Lautsprechern auch an verschiedenen Positionen, sodass dieses Gehäuse 
			zusätzlich in seiner Position variabel positionierbar sein muss. Einige Lautsprecher mit stärkeren Verstärkern haben zusätzlich Kühlkörper an der Lautsprecherrückseite, 
			aus diesem Grund und der Tatsache, dass eventuell weitere Regler (Phase, TP, EQ) vorhanden sind, ist ein gewisser Abstand zwischen den Gehäusen notwendig.
			Auch die Positionen der Endanschläge sind nicht überall gleich, das bedeutet, dass auch die Endschalter in ihrer Position variierbar sein müssen.
		\label{sec:Theorie Mecanik}
	\chapter{Praxis-Teil}
		\section{verwendetes Material?}
			\subsection{der Arduino Nano}
			\subsection{die Arduino IDE}
			\subsection{FreeCad}
			\subsection{Cura}
			\subsection{Eagle}
			\subsection{der Schrittmotor ?}
		\section{Realisierung der Schaltung}
			\subsection{Motorsteuerung}
			\subsection{IR-Receiver}
			\subsection{Display}
			\subsection{Anschlagerkennung durch Strommessung}
			\subsection{Anschlagerkennung durch Endschalter}
			\subsection{Gesamtschaltung}
		\section{Software}
		\section{Das Gahäuse}
			\subsection{Motorgehäuse}
			\subsection{Displaygehäuse}
			\subsection{Gehäuseständer}
		\section{Inbetriebnahme}
	\chapter{Anhänge}
		\section{Tabelle Teileübersicht}
		\section{Datenblätter}
			Eine Tabelle mit der Auflistung aller Einzelteile mit anschließenden wichtigen Bereichen der Datenblätter.
			Die gesamten Datenblätter werden nicht eingefügt, jedoch ein Link der zu einem derzeitigen Datenblatt im Internet führt.
		\section{Schaltplan}
			Die 2 Schaltpläne der eigenen Platinen.
		\section{Layout}
			Die 2 Platinenlayouts.
		\section{Abmessungen}
			Technische Zeichnungen von FreeCAD mit den Abmessungen der Gehäuse.
		\section{Quellcode}
			Ausschnitte des Quellcodes.
	
\end{document}