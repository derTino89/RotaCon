\documentclass[a4paper, twoside, 10pt]{article}

\usepackage[T1]{fontenc}
\usepackage[utf8]{inputenc}
\usepackage{lmodern}
\usepackage[ngerman]{babel}
\usepackage{titling}

\pretitle{\begin{center}\linespread{1.5}\huge\sffamily}
\posttitle{\par\end{center}\vspace{0.5em}}
\title{
	\Huge Dokumentation RotaCon v.01A
	\\
	\Large Projektarbeit
	\\
	im Studiengang Medientechnik}
\author{Tino Liebenow}
\date{Wintersemester 19/20}

\begin{document}
	\maketitle
	\newpage
	\tableofcontents
	\newpage
	\addcontentsline{toc}{section}{Abbildungsverzeichnis}
	\section*{Abbildungsverzeichnis}
	\newpage
	\addcontentsline{toc}{section}{Tabellenverzeichnis}
	\section*{Tabellenverzeichnis}
	\newpage
	\section{Überblick}
		Überblick über den Aufbau der Dokumentation
	\section{Ziel}
		was möchte Tino eigentlich machen?
	\section{Stand der Technik}	
		gibts sowas schon?
	\section{Theorie-Teil}
		\subsection{Theorie Hardware}
		\subsection{Tehorie Software}
		\subsection{Tehorie Mechanik}
	\section{Praxis-Teil}
		\subsection{verwendetes Material?}
			\subsubsection{der Arduino Nano}
			\subsubsection{die Arduino IDE}
			\subsubsection{FreeCad}
			\subsubsection{Cura}
			\subsubsection{Eagle}
			\subsubsection{der Schrittmotor ?}
		\subsection{Realisierung der Schaltung}
			\subsubsection{Motorsteuerung}
			\subsubsection{IR-Receiver}
			\subsubsection{Display}
			\subsubsection{Anschlagerkennung durch Strommessung}
			\subsubsection{Anschlagerkennung durch Endschalter}
			\subsubsection{Gesamtschaltung}
		\subsection{Software}
		\subsection{Das Gahäuse}
			\subsubsection{Motorgehäuse}
			\subsubsection{Displaygehäuse}
			\subsubsection{Gehäuseständer}
		\subsection{Inbetriebnahme}
	\section{Anhänge}
		\subsection{Tabelle Teileübersicht}
		\subsection{Datenblätter}
			Eine Tabelle mit der Auflistung aller Einzelteile mit anschließenden wichtigen Bereichen der Datenblätter.
			Die gesamten Datenblätter werden nicht eingefügt, jedoch ein Link der zu einem derzeitigen Datenblatt im Internet führt.
		\subsection{Schaltplan}
			Die 2 Schaltpläne der eigenen Platinen.
		\subsection{Layout}
			Die 2 Platinenlayouts.
		\subsection{Abmessungen}
			Technische Zeichnungen von FreeCAD mit den Abmessungen der Gehäuse.
		\subsection{Quellcode}
			Ausschnitte des Quellcodes.
	
\end{document}